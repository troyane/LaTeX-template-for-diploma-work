%= = = = = = = = = = = = = = = = = = = = = = = = = = = = = = = = = = = = = = = = = = =
%       Шаблон розроблений Герасимчуком Назаром (nazar.gerasymchuk@gmail.com)
%   доступний за адресою: https://github.com/troyane/LaTeX-template-for-diploma-work
%- - - - - - - - - - - - - - - - - - - - - - - - - - - - - - - - - - - - - - - - - - - 
%  Файл першої частини chapter1.tex
%= = = = = = = = = = = = = = = = = = = = = = = = = = = = = = = = = = = = = = = = = = =

\newpage
\chapter{\textsc{Основні\- поняття\- з життя жаби}}

\section{Походження}
Найдавнішим викопним безхвостими земноводними є протобатрахус (Protobatrachus massinoti) з тріасу Мадагаскару (який ще мав хвіст). Зоологи з Манчестерського університету виявили в Коста-Ріка, у заповіднику вологих тропічних лісів Монтеверде, самицю маленької деревної жаби виду Іsthmohyla rіvularіs, що вважався вимерлим ще 20 років тому. Вчені зазначають, що іще в 2007 році в Коста-Ріка була помічена чоловіча особина жаби цього виду. А виявлення тепер жіночої особини дозволяє вченим припустити, що ці земноводні розмножуються й здатні вижити.

Спочатку дослідникам вдалося знайти самця жаби, самостійно наслідуючи звук жаби, а потім голова заказника Тропічного наукового центру Монтеверде Луїс Обандо виявив і маленьку самицю жаби, що сиділа на листі. Герпетолог музею манчестерского університету Ендрю Грій заявив, що виявлення цієї жаби є вершиною всієї його кар'єри.

<<Зараз, коли ми знаємо, що обидві статі цього виду жаби існують у дикій природі, важливо дослідити особливості існування цього виду й докласти максимум зусиль для його збереження>>, - сказав Грій.

Він також повідомив, що виявлена самиця жаби, довжина тіла якої 2,5 сантиметри, була коричневою з маленькими зеленим плямами. Знайти її було вкрай важко, тому що чоловічі особини часто видають голосні закличні звуки, у той час як самки роблять це досить рідко.

\section{Поширення}
Поширені безхвості земноводні в усіх ландшафтно-географічних зонах, крім полярних областей (трав'яна жаба заходить і за Полярне коло) та безводних пустель. Найбільше безхвостих земноводних у Тропічній Америці. В Україні трапляються 12 видів.


\section{Класифікація}
Ряд налічує 256 родів та близько 3 500 видів, 16—31 родин: піпові — Pipidae (підродина пазуристі жаби, піпи), круглоязикові, часничниці, дереволази, квакші, рінодерми, свистуни, ропухові, вузькороті жаби,Скляні жаби та інші.

