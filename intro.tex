%= = = = = = = = = = = = = = = = = = = = = = = = = = = = = = = = = = = = = = = = = = =
%       Шаблон розроблений Герасимчуком Назаром (nazar.gerasymchuk@gmail.com)
%   доступний за адресою: https://github.com/troyane/LaTeX-template-for-diploma-work
%- - - - - - - - - - - - - - - - - - - - - - - - - - - - - - - - - - - - - - - - - - - 
%  Файл вступу gloss.tex
%= = = = = = = = = = = = = = = = = = = = = = = = = = = = = = = = = = = = = = = = = = =


\newpage
\chapter*{\textsc{Вступ}}
\addcontentsline{toc}{chapter}{\textsc{Вступ}}

Життя жаби звичайної -- нелагка штука, з огляду на атеїстичні погляди нашого суспільства. Як писав один із дослідників жаби, Піонер Гриша: <<Жаба живе в саду, живиться продуктами народного господарства, тому жаба – шкідник, її треба знищувати. Наш народ вже даво оголосив війну жабі>>.
 
\textbf{Мета дослідження} -- дослідити життя жаби глибше.

Поставлена мета передбачає виконання таких \textbf{завдань:}

\begin{enumerate}
	\item Вивчення літератури про жаб та атеїзм.
	\item Дослідити ставлення Дарвіна до жаб.
	\item Отримання та аналіз результатів досліджень.
\end{enumerate}