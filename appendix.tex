%= = = = = = = = = = = = = = = = = = = = = = = = = = = = = = = = = = = = = = = = = = =
%       Шаблон розроблений Герасимчуком Назаром (nazar.gerasymchuk@gmail.com)
%   доступний за адресою: https://github.com/troyane/LaTeX-template-for-diploma-work
%- - - - - - - - - - - - - - - - - - - - - - - - - - - - - - - - - - - - - - - - - - - 
%  Файл додатків appendix.tex
%= = = = = = = = = = = = = = = = = = = = = = = = = = = = = = = = = = = = = = = = = = =


\chapter*{\textsc{Додатки}}
\addcontentsline{toc}{chapter}{\textsc{Додатки}}

\subsection*{Частини програмного коду жаби-кіборга}
Вихідний код модуля myTypes.h, в якому описано основну структуру -- точку в тривимірному просторі glPoint3d.
\begin{verbatim}
#include <math.h>
#include <QtOpenGL>
// опис "нуля" термінами OpenGL
#define GL_0 (GLfloat(0.0))

const float EPS = 1.0E-5;		// похибкa 
const float RTOD = 180/M_PI;	// коеф. радіани->градуси

using namespace std;

//ф-ія квадрату
inline GLfloat sqr(GLfloat x){ 
   return (GLfloat) (x*x); 
}  

// ф-ія перевірки x на близкість до 0.0
inline bool isZero(GLfloat x){ 
   return fabsf((float)x)<EPS ? true : false; 
}

class glPoint3d{
// клас точки в 3d з текстурними координатами
public:
   GLfloat x, y, z;
   GLfloat tx, ty;

   glPoint3d(GLfloat nx, GLfloat ny, GLfloat nz, GLfloat ntx, 
   				GLfloat nty){
    x=nx; y=ny; z=nz; tx=ntx; ty=nty;
};  
// занулення
void makeZeros(){x=GL_0; y=GL_0; z=GL_0; tx=GL_0; ty=GL_0;}
// нормування
void normalize(){
   GLfloat l = 1.0f/sqrt(sqr(x) + sqr(y) + sqr(z));
   if(!isZero(l)){
    x *= l;
    y *= l;
    z *= l;
   }else{ x=GL_0; y=GL_0; z=GL_0; }
};
// пошук відстані до other
GLfloat to(glPoint3d *other){ 
   return (GLfloat) sqrt(sqr(x-other->x) + 
            sqr(y-other->y)+sqr(z-other->z)); 
  };
};
\end{verbatim}